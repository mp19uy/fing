\documentclass[a4paper,10pt]{article}
%\documentclass[a4paper,10pt]{scrartcl}
\usepackage[utf8]{inputenc}
\usepackage{hyperref}
\usepackage{url}
\usepackage[margin=1in]{geometry}
\usepackage{underscore}
\usepackage{amsfonts}
\usepackage{amsmath}
\usepackage[normalem]{ulem}
\usepackage{cancel}
\title{Titulo}
\author{Autor}
\date{Fecha}
\newcommand{\dstr}[1]{\mathbb{#1}}
\newcommand{\RR}{\dstr{R}}
\newcommand{\CC}{\dstr{C}}
\newcommand{\bd}[1]{\mathbf{#1}}
\pdfinfo{%
  /Title    ()
  /Author   ()
  /Creator  ()
  /Producer ()
  /Subject  ()
  /Keywords ()
}

\begin{document}
\section*{Multiple Opción}
\subsection*{Ejercicio 1}
Sean $a \in \RR$ y $z_0 \in \CC$.\\
Si $z=x+iy$, que representa la ecuación $z\bar{z} + z_0 \bar{z} + \bar{z_0} z + a = 0$ en el plano $xy$?\\
 \\
Considero $z_0 = b + ic$, luego:\\\\
$\begin{array}{ccccccccc}
z\bar{z} & + & z_0 \bar{z} & + & \bar{z_0} z & + & a & = & 0 \\
x^2 + y^2 & + & (b+ic)(x-iy) & + & (b-ic)(x+iy) & + & a & = & 0 \\
x^2 + y^2 & + & bx + cy + i(cx - by) & + & bx + cy - i(cx - by) & + & a & = & 0\\
x^2 + y^2 & + & bx + cy\ \cancel{+ i(cx - by)} & + & bx + cy\ \cancel{- i(cx - by)} & + & a & = & 0
\end{array}$\\\\\\
$\Rightarrow x^2 + y^2 + 2bx + 2cy + a = 0$\\\\
Recordando la ecuación de la circunferencia:\\ 
$x^2 + y^2 + Ax + By + C = 0\ (A,B,C \in \RR)$, 
y $\left\{ \begin{array}{l}
  \displaystyle \bd{Centro} = \left(-\frac{A}{2}, -\frac{B}{2} \right)  \\
   \\
   \displaystyle \bd{Radio} = \sqrt{\frac{A^2}{4} + \frac{B^2}{4} - C}
  \end{array} \right.$\\
  Tenemos $A = 2b$, $B = 2c$ y $C = a$ y:\\
  $\displaystyle \sqrt{\frac{A^2}{4} + \frac{B^2}{4} - C} = \sqrt{\frac{{(2b)}^2}{4} + \frac{{(2c)}^2}{4} - a} = \sqrt{\frac{4b^2}{4} + \frac{4c^2}{4} - a} = \sqrt{b^2 + c^2 - a} = \sqrt{{|z_0|}^2 - a}$\\\\
Entonces es una circunferencia si ${|z_0|}^2 - a > 0 \Leftrightarrow {|z_0|}^2 > a$

\subsection*{Ejercicio 2}
$a_n = \displaystyle \int_0^{\frac{\pi}{4}}\frac{cos x - sen x}{(sen x + cos x)^{n+1}} dx$  y  $b_n = \displaystyle \frac{1}{n} - a_n$, con $n \in \RR$\\
(I) $\displaystyle \sum_{n=1}^{+\infty}a_n$ y (II) $\displaystyle \sum_{n=1}^{+\infty}b_n$, estudiar convergencia de la series.\\\\\\
Haciendo cambio de variable:\\
$u = sen x + cos x \\
du = cos x - sen x$\\\\
Extremos:\\
$sen(0) + cos(0) = 1 \\
sen(\frac{\pi}{4}) + cos(\frac{\pi}{4}) = \sqrt{2}$\\\\

$\displaystyle \int_1^{\sqrt{2}}\frac{1}{u^{n+1}} dx = \frac{-1}{n u^n} \bigg|_{1}^{\sqrt{2}} = \frac{-1}{n \sqrt{2}^n} + \frac{1}{n 1^n} = \fbox{$\displaystyle \frac{1}{n} - \frac{1}{n \sqrt{2}^n} = a_n$}$ y \fbox{$\displaystyle b_n = \frac{1}{n \sqrt{2}^n}$}\\\\
Aplicamos el criterio del cociente para series sobre $b_n$, el cual dice:\\
$a_n > 0$\\
Si para $n > n_0$ $\left\{ \begin{array}{ll} \frac{a_{n+1}}{a_n} \leq k < 1 & \Rightarrow \sum a_n \text{ converge} \\ \frac{a_{n+1}}{a_n} \geq 1 & \Rightarrow \sum a_n \text{ diverge} \end{array} \right. $\\\\
$ \displaystyle \lim_{n\rightarrow\infty}\frac{\frac{1}{(n+1) \sqrt{2}^{n+1}}}{\frac{1}{n \sqrt{2}^n}} = \lim_{n\rightarrow \infty}\frac{n \sqrt{2}^n}{(n+1) \sqrt{2}^{n+1}} = \lim_{n\rightarrow\infty} \frac{n}{(n+1)\sqrt{2}} = \lim_{n\rightarrow\infty} \frac{n}{\sqrt{2}n + \sqrt{2}} = \frac{1}{\sqrt{2}} < 1$\\\\
$\Rightarrow b_n $ converge\\\\
Se sabe que $\displaystyle \sum_{n=1}^{+\infty} \frac{1}{n}$ diverge, entonces si: $\displaystyle a_n = \underbrace{\frac{1}{n}}_\text{Diverge} - \underbrace{\frac{1}{n \sqrt{2}^n}}_\text{Converge}$\\
Y la suma de algo que diverge menos algo que converge, evidentemente sigue divergiendo.\\
$\Rightarrow a_n$ diverge.

\subsection*{Ejercicio 4}
Sea $\displaystyle F(x) = \int_{\pi}^{x} \frac{e^t}{3+sen(t)} dt$, cual es el valor de $(F^{-1})'(0)$?\\\\
Se cumple que $\displaystyle (F^{-1})'(0) = \frac{1}{F'(F^{-1}(0))}$\\
Y $F^{-1}(0) = A \Leftrightarrow F(A) = 0$, y esto sucede cuando $x = \pi$ en $F(x)$ ya que:\\
$\displaystyle F(\pi) = \int_{\pi}^{\pi} \frac{e^t}{3+sen(t)} dt = 0$, puesto que una integral evaluada entre $\pi$ y $\pi$ vale 0.\\\\
Entonces $F^{-1}(0) = A = \pi$  y ahora solo resta evalular $F'(F^{-1}(0)) = F'(\pi)$.\\
F'(x) por ser $\frac{e^t}{3+sen(t)}$ continua (composición de funciones continuas) $\forall x \in \RR$, entonces puedo aplicar el Teorema Fundamental del Cálculo que dice que $\displaystyle F'(x) = \frac{e^x}{3+sen(x)}$.\\
$\displaystyle F'(\pi) = \frac{e^\pi}{3+sen(\pi)} = \frac{e^\pi}{3}$ y finalmente $(F^{-1})'(0) = \displaystyle \frac{1}{ \frac{e^\pi}{3}} = \frac{3}{e^\pi} = 3 e^{-\pi}$
\end{document}
