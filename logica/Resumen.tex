%%%%%%%%%%%%%%%%%%%%%%%%%%%%%%%%%%%%%%%%%%%%%%%%%%%%%%%%%%%%%%%%%%%%%%%%%%%%%
% Tarea 2 : Parte 2 - Programacion 3
%%%%%%%%%%%%%%%%%%%%%%%%%%%%%%%%%%%%%%%%%%%%%%%%%%%%%%%%%%%%%%%%%%%%%%%%%%%%%
\documentclass[article,12pt]{report}
% This first part of the file is called the PREAMBLE. It includes
% customizations and command definitions. The preamble is everything
% between \documentclass and \begin{document}.
\newcommand{\margin}{2cm}
\usepackage[top=\margin,right=\margin,left=\margin,bottom=\margin]{geometry}
%%%%%%%%%%%%%%%%%%%%%%%%%%%%%%%%%%%%%%%%%%%%%%%%%%%
% Another way to Set the margins
%%%%%%%%%%%%%%%%%%%%%%%%%%%%%%%%%%%%%%%%%%%%%%%%%%%
%\usepackage[margin=2cm]{geometry} % set the %margins to 2cm on all sides
%%%%%%%%%%%%%%%%%%%%%%%%%%%%%%%%%%%%%%%%%%%%%%%%%%%
\usepackage{framed}
\usepackage{array}
\usepackage{enumerate}
% to use urls
\usepackage{hyperref}
% to include figures
\usepackage{graphicx}
% great math stuff
\usepackage{amsmath}
% for blackboard bold, etc
\usepackage{amsfonts}
% better theorem environments
\usepackage{amsthm}
% para tildes y eñes y otras yerbas
\usepackage[utf8]{inputenc}
\usepackage[T1]{fontenc}
\usepackage{lmodern}
\usepackage{tikz}
% para links y emails
\usepackage{hyperref}
% Use fancyhdr to define our own headers
\usepackage{fancyhdr}
\setlength{\headheight}{25pt}
% Keeps LaTeX happy, takes care of some warnings
\pagestyle{fancy}
% Multi rows
\usepackage{multirow}
% caracteres
\usepackage{textcomp}
% various theorems, numbered by section
\newtheorem{thm}{Theorem}[section]
\newtheorem{lem}[thm]{Lemma}
\newtheorem{prop}[thm]{Proposition}
\newtheorem{cor}[thm]{Corollary}
\newtheorem{conj}[thm]{Conjecture}
\DeclareMathOperator{\id}{id}
% for bolding symbols
\newcommand{\bd}[1]{\mathbf{#1}}
% for Real numbers
\newcommand{\RR}{\mathbb{R}}
\newcommand{\NN}{\mathbb{N}}
% for Integers
\newcommand{\ZZ}{\mathbb{Z}}
\newcommand{\col}[1]{\left[\begin{matrix} #1 \end{matrix} \right]}
\newcommand{\comb}[2]{\binom{#1^2 + #2^2}{#1+#2}}
% Definitions to fill the header with
%%%%%%%%%%%%%%%%%%%%%%%%%%%%%%%%%%%%%%%%%%%%%%%%%%%%%%%%%%%%%%%%%%%%%%%%%%%%%%%%
\newcommand{\grupo}{Resumen Teórico de Lógica}
\newcommand{\nombres}{Realizado por Martín Pacheco}
\newcommand{\semestre}{}
\newcommand{\materia}{Facultad de Ingeniería}
\newcommand{\tarea}{Resumen Teórico}
\newcommand{\version}{}
%%%%%%%%%%%%%%%%%%%%%%%%%%%%%%%%%%%%%%%%%%%%%%%%%%%%%%%%%%%%%%%%%%%%%%%%%%%%%%%%
% Redefinition of commands
\newcommand{\sectionh}[1]{\line(1,0){500}\section*{#1}}
% Now custimize the header. Make the text bold.
% We'll get something like:
%
% 123456789 LaTeX 101
% J. Random Student Assignment N Today's Date
% --------------------------------------------------
%
% This layout is pretty simple, and should be enough for an assignment
% If you want more, you can consult the documentation
% http://www.ctan.org/tex-archive/macros/latex/contrib/fancyhdr/fancyhdr.pdf
\chead{\textbf{\small \grupo\\ \nombres}}

% Here is an example for customising the numbering
% It changes the first level of numbering to bolded (a), (b), (c), etc
%\renewcommand{\theenumi}{\textbf{(\alph{enumi})}}
%\renewcommand{\labelenumi}{\theenumi}
% Other options to play with are to change \theenumii, \labelenumii, and enumii for the second level of nesting,
% and so on to \theenumiv, \labelenumiv, and enumiv for the fourth level of nesting.
% The possible formats are \arabic (1, 2...), \alph (a, b...), \Alph (A, B...), \roman (i, ii...), and \Roman (I, II...)
{\setlength{\tabcolsep}{0.7em}

\usepackage{picture}
% Here begin the document
\begin{document}
\subsubsection*{Alfabeto}
Conjunto dado de simbolos. Ejemplo: $\{a, b\}$
\subsubsection*{Lenguaje}
Conjunto de frases o palabras (tiras, secuencias, strings, etc) construidas sobre un \textbf{Alfabeto} dado.
\subsubsection*{$\bd{\{a, b\}^*}$}
Se define inductivamente por las siguientes clausulas:
\begin{enumerate}[i.]
  \item $\varepsilon \in \{a,b\}^*$
  \item Si $w \in \{a,b\}^*$, entonces $aw \in \{a,b\}^*$
  \item Si $w \in \{a,b\}^*$, entonces $bw \in \{a,b\}^*$
\end{enumerate}

\subsubsection*{$\NN$}
Definicion inductiva de los naturales:
\begin{enumerate}[i.]
  \item $0 \in \NN$
  \item Si $n \in \NN$, entonces $S(n) \in \NN$
\end{enumerate}
(Siendo $S(n)$ el sucesor de n)

\subsubsection*{P.I.P. (Principio de Inducción Primitiva) para $\NN$}
\begin{itemize}
  \item Sea $ \bd{P} $ una propiedad sobre los elementos de $\NN$ que cumple:
  \begin{enumerate}[i.]
    \item $P(0)$ se cumple
    \item Si $P(n)$ se cumple, entonces $P(S(n))$ se cumple
  \end{enumerate}
  \item Entonces $\bd{P}$ se cumple $\forall n \in \NN$
\end{itemize}

\subsubsection*{E.R.P. (Esquema de Recursión Primitiva) para $\NN$}
\begin{itemize}
  \item Sea $\bd{B}$ un conjunto y sean:
  \begin{itemize}
    \item $f_0 \in B$
    \item $f_S: \NN\times B\rightarrow B$
  \end{itemize}
  \item Entonces existe una unica funcion $F:\NN\rightarrow B$ que:
  \begin{enumerate}[i.]
    \item $F(0) = f_0$
    \item $F(S(n))= f_S(n,F(n))$
  \end{enumerate}
\end{itemize}

\subsubsection*{PROP (Def 1.1.2)}
Es el conjunto definido inductivamente por:
\begin{enumerate}[i)]
  \item $p_i \in PROP, \forall i \in \NN$
  \makebox(0,0){\put(0,-3\normalbaselineskip){%
               $\left.\rule{0pt}{2\normalbaselineskip}\right\}$ formulas atómicas}}
  \item $\bot \in PROP$
  \item Si $\alpha \in PROP$ y $\beta \in PROP$, entonces $(\alpha \qedsymbol \beta)$\\
  $\qedsymbol = \{\wedge, \vee, \rightarrow, \leftrightarrow \} = C$ ( esta notación será utilizada mas adelante)
  \item Si $\alpha \in PROP$, entonces $(\neg \alpha) \in PROP$
\end{enumerate}
Los objetos de $PROP$ se llaman: \textbf{Formulas Proposicionales}

\subsubsection*{$\bd{\sum_{PROP}} (Def 1.1.1)$}
El alfabeto de la lógica proposicional es el conjunto $\sum_{PROP}$ que consiste de:
\begin{enumerate}[i)]
  \item $[$letras de proposicion$]$  $P_i, \forall i \in \NN$
  \item $[$conectivos$]$ $\bot, \wedge, \vee, \rightarrow, \leftrightarrow$
  \item $[$simbolos auxiliares$]$ $(\ ,\ )$
\end{enumerate}

\subsubsection*{Notación}
C $= \{\wedge, \vee, \rightarrow, \leftrightarrow \}$\\
AT $= \{\bot, p_0, p_1, ...\}$\\
P $= \{ p_0, p_1, ...\}$

\subsubsection*{P.I.P. (Principio de Inducción Primitiva) para $PROP$ (Teo 1.1.3)}
Sea $\bd{P}$ una propiedad sobre $PROP$ que cumple:
\begin{itemize}
  \item[PB)] Paso Base
  \begin{itemize}
    \item $P(p_i), \forall i \in \NN$
    \item $P(\bot)$
  \end{itemize}
  \item[PI)] Paso Inductivo
  \begin{itemize}
    \item Si $P(\alpha)$ y $P(\beta)$ entonces $P((\alpha \qedsymbol \beta))$, $\qedsymbol \in $ C
    \item Si $P(\alpha)$ entonces $P((\neg \alpha))$
  \end{itemize}
\end{itemize}
Entonces $P(\alpha) \in PROP$
\end{document}
