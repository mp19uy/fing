\documentclass[12pt,a4paper]{article}

\usepackage[utf8]{inputenc}
\usepackage{cancel}
\usepackage{ntheorem}
\usepackage{amsmath}
\usepackage{amsfonts}
\newtheorem*{definition}{Definición}
\newtheorem*{props}{Propiedades}
%Margenes

%opening
\title{Todo lo que querias saber sobre el Ejercicio 4 pero nunca te animaste a preguntar}
\author{Resumen de Martin Pacheco, digitalizado por Mauricio Irace}

\begin{document}
\maketitle
(En todas las secciones se asumira $\Gamma \subseteq PROP$)
% RECORDAR \models
% \vdash

\section*{Definiciones}

\begin{definition} \emph{CONS}\\
	$CONS( \Gamma ) := \{ \varphi \in PROP : \Gamma \vdash  \varphi \}$
\end{definition}
\begin{definition}

\emph{Teoría}\\
	$\Gamma$ es Teoría $ \Leftrightarrow CONS(\Gamma)\subseteq \Gamma$ \\ 
	 (Si $ \Gamma \vdash \varphi$ entonces $\varphi \in \Gamma$)
\end{definition}

\begin{definition}
	\emph{Consistente}\\
	$\Gamma$ es Consistente $ \Leftrightarrow \bot \in CONS(\Gamma)$ \\
	($ \Gamma \cancel{\vdash} \bot$)
\end{definition}

\begin{definition}
	\emph{Consistente Maximal}\\
	$\Gamma$ es Consistente Maximal $\Leftrightarrow$
	$\left\{\begin{array}{l}
		 \text{\textbullet}\ \ \Gamma \textnormal{ es Consistente} \\ 
     \text{\textbullet}\ \ \left\{ \begin{array}{l}
      \overline{\forall} \Delta \textnormal{ Consistente: } ( \Gamma \subseteq \Delta \Rightarrow \Gamma = \Delta ) \\
     \Leftrightarrow \textnormal{ (otra interpretación equivalente)}\\
		 \overline{\forall} \Delta \textnormal{, Si } \Gamma \subset \Delta \Rightarrow \Delta \textnormal{ es Inconsistente} \\
		 \textnormal{ (Notar $\subset$ es Contenido Estricto) }
     \end{array} 
		 \right.
	 \end{array}
	 \right.$

\end{definition}

\begin{definition}
\emph{Completo} \\
$\Gamma$ es Completo $\Leftrightarrow$ $\left\{ \begin{array}{l} \text{\textbullet}\ \ \Gamma \textnormal{es Consistente} \\
 \text{\textbullet}\ \  \overline{\forall} \varphi \in PROP \textnormal{ o bien } \Gamma \vdash \varphi \textnormal{ o } \Gamma \vdash \neg \varphi
  \end{array}
  \right.$
\end{definition}

\section*{Propiedades}

\begin{props} CONS \\

	\begin{itemize}
		\item $ CONS( \emptyset) = \{ \varphi \in  PROP :$ $ \vdash \varphi  \}$ (Teoremas)
		\item $ CONS(\{ \bot \}) = PROP$
		\item $ A \subseteq B \Rightarrow CONS(A) \subseteq CONS(B)$
		\item $CONS(CONS(\Gamma)) = CONS(\gamma)$
		\item $CONS(\Gamma)$ es Teoría (se deduce de lo anterior)
		\item $\Gamma \subseteq CONS(\Gamma)$
	\end{itemize}
\end{props}

\begin{props} Teoría \\

	\begin{itemize}
		\item 	$\Gamma $ es Teoría $\Rightarrow \Gamma = CONS(\Gamma)$
	\end{itemize}
\end{props}

\begin{props} Consistente\\
	Recordando que $v(\Gamma) = 1\Leftrightarrow(\overline{\forall}\varphi \in \Gamma)v(\varphi)=1$
	\begin{itemize}
		\item $\Gamma$ es Consistente $\Leftrightarrow (\overline{\exists}v :$ Valuación $)v(\Gamma)=1$ 
	\end{itemize}

\end{props}
	
\begin{props}Inconsistente\\

	\begin{itemize}
		\item$\Gamma$ es Inconsistente $\Leftrightarrow \bot \in CONS(\Gamma) \Leftrightarrow \Gamma \vdash \bot$\\
		$\Leftrightarrow \overline{\forall}\varphi \in PROP$, $\Gamma \vdash \varphi$\\
		$\Leftrightarrow \overline{\exists}\varphi \in PROP $, $\Gamma \vdash \varphi$ y $\Gamma \vdash \neg\varphi$
		\item $\Gamma \bigcup \{\varphi\}$ es Inconsistente $\Rightarrow \Gamma \vdash \neg \varphi$		
	\end{itemize}

\end{props}
\begin{props}Consistente Maximal\\

	\begin{itemize}
		\item $\Gamma$ es Consistente Maximal (CM) entonces:
			\begin{itemize}
				\item $\Gamma$ es Teoría
				\item $\varphi \in \Gamma \Leftrightarrow \neg \varphi \cancel{\in} \Gamma$
				\item $\overline{\forall}\varphi \in PROP, \varphi \in \Gamma$ o $\neg\varphi \in \Gamma$ (Pero no ambos)
				
			\end{itemize}
		\item $\Gamma$ CM $\Leftrightarrow \Gamma$ es Teoría y ($\overline{\exists}! v : $Valuación)$v(\Gamma) = 1$
	\end{itemize}
\begin{props}Completo\\
	$\Gamma$ es Completo Sii
	\begin{itemize}
		\item $CONS(\Gamma)$ es CM
		\item $\overline{\exists}! v: $ Valuación $v(\Gamma) = 1$
	\end{itemize}
\end{props}
\end{props}	
\end{document}
