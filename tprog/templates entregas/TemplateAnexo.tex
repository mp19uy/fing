\documentclass[10pt,spanish]{article}  
\usepackage[spanish]{babel} %Indica que escribiermos en español
\usepackage[utf8]{inputenc} %Indica qué codificación se está usando ISO-8859-1(latin1)  o utf8 
\selectlanguage{spanish}

%%%%%%% DEFINICIÓN DE VARIABLES %%%%%%%%%%%%%%%%%%%
\newcommand{\Tarea}{Tarea \{ \# \}}
\newcommand{\Titulo}{Anexo}
\newcommand{\Materia}{\{ MATERIA \}}
\newcommand{\Curso}{Curso \{ AÑO \}}
\newcommand{\Grupo}{Grupo \{ NRO \}}
% Nombre y Cedulas de los integrantes del grupo
\newcommand{\NomIntegrUno}{Integrante 1}
\newcommand{\CiIntegrUno}{CI}
\newcommand{\NomIntegrDos}{Integrante 2}
\newcommand{\CiIntegrDos}{CI}
\newcommand{\NomIntegrTres}{Integrante 3}
\newcommand{\CiIntegrTres}{CI}
\newcommand{\NomIntegrCuatro}{Integrante 4}
\newcommand{\CiIntegrCuatro}{CI}
\newcommand{\NomIntegrCinco}{Integrante 5}
\newcommand{\CiIntegrCinco}{CI}
% Nombre del Docente
\newcommand{\NomDocente}{Nombre Docente}
% Texto del Encabezado
\newcommand{\EncabezadoIzq}{Universidad de la República}
\newcommand{\EncabezadoMed}{Facultad de Ingeniería}
\newcommand{\EncabezadoDer}{Instituto de Computación}

%%%%%%%% PREÁMBULO %%%%%%%%%%%%

\usepackage{amsmath} % Comandos extras para matemáticas (cajas para ecuaciones,
% etc)
\usepackage{amssymb} % Simbolos matematicos (por lo tanto)
\usepackage{graphicx} % Incluir imágenes en LaTeX
\usepackage{color} % Para colorear texto
\usepackage{subfigure} % subfiguras
\usepackage{float} %Podemos usar el especificador [H] en las figuras para que se queden donde queramos
\usepackage{capt-of} % Permite usar etiquetas fuera de elementos flotantes
% (etiquetas de figuras)
\usepackage{sidecap} % Para poner el texto de las imágenes al lado
	\sidecaptionvpos{figure}{c} % Para que el texto se alinie al centro vertical
\usepackage{caption} % Para poder quitar numeracion de figuras
\usepackage{commath} % funcionalidades extras para diferenciales, integrales,
% etc (\od, \dif, etc)
\usepackage{cancel} % para cancelar expresiones (\cancelto{0}{x})


% MARGENES 
\usepackage{anysize} 					% Para personalizar el ancho de  los márgenes
\marginsize{2cm}{2cm}{2cm}{2cm} % Izquierda, derecha, arriba, abajo
\setlength{\parindent}{0cm}
\setlength{\parskip}{\baselineskip}

% Para que las referencias sean hipervínculos a las figuras o ecuaciones aparezcan en color
\usepackage[colorlinks=true,plainpages=true,citecolor=blue,linkcolor=blue]{hyperref}
%\usepackage{hyperref} 

% ENCABEZADO Y PIE DE PÁGINA
\usepackage{fancyhdr} 
\pagestyle{fancy}
\fancyhf{}

\fancyhead[L]{\EncabezadoIzq} %encabezado izquierda
\fancyhead[C]{\EncabezadoMed} %encabezado centro
\fancyhead[R]{\EncabezadoDer}   % dereecha

\fancyfoot[R]{\Curso}  % Pie derecha
\fancyfoot[C]{\thepage}  % centro
\fancyfoot[L]{\Tarea}  %izquierda
\renewcommand{\footrulewidth}{0.4pt}

\usepackage[framemethod=tikz]{mdframed}
\newmdenv[
  topline=false, %oculta linea arriba
  bottomline=false, % oculta linea abajo
  rightline=false, % oculta linea derecha
  skipabove=\topsep,
  skipbelow=\topsep,
]{siderules}

\numberwithin{figure}{section} % numeración de figuras por seccion
\usepackage[font=bf,labelfont=bf]{caption} % setea estilo bf (bold) a los caption (texto y label)

\usepackage[compact]{titlesec} %Formato de las secciones
\titleformat*{\section}{\LARGE\bfseries}
\titleformat*{\subsection}{\Large\bfseries}
\titleformat*{\subsubsection}{\large\bfseries}


\usepackage{etoolbox} % Agrega puntos a las sections en el TOC
\makeatletter
\patchcmd{\l@section}
  {\hfil}
  {\leaders\hbox{\normalfont$\m@th\mkern \@dotsep mu\hbox{.}\mkern \@dotsep mu$}\hfill}
  {}{}
\makeatother

\title{\Tarea - \Titulo}

%%%%%%%% TERMINA PREÁMBULO %%%%%%%%%%%%

\begin{document}

%%%%%%%%%%%%%%%%%%%%%%%%%%%%%%%%%% PORTADA %%%%%%%%%%%%%%%%%%%%%%%%%%%%%%%%%%%%%%%%%%%%
\begin{minipage}{0.48\textwidth} \begin{flushleft}
\end{flushleft}\end{minipage}

%%%
\begin{center}																		%%%
\newcommand{\HRule}{\rule{\linewidth}{0.5mm}}									%%%\left
 																					%%%


													 								%%%
\vspace*{1cm}								%%%
																				%%%	
\textsc{\huge \Materia}\\[1.5cm]	

\textsc{\huge \Tarea				%%%
}\\[1.5cm]													%%%

    																				%%%
\vspace*{1cm}																		%%%
																					%%%
\HRule \\[0.4cm]																	%%%
{ \huge \bfseries \Titulo}\\[0.3cm]	%%%
 																					%%%
\HRule \\[4cm]																	%%%
 																				%%%
																					%%%
\begin{minipage}{0.8\textwidth}													%%%
\begin{flushleft} \large															%%%
\textsc{\LARGE \Grupo}\\
\LARGE{\textbf{Integrantes}}\\	
\Large
\vspace{0.3cm}
  \begin{tabular}{ | p{10cm} | p{2.5cm} | }
    \hline
    \textbf{Nombre} & \textbf{CI} \\ \hline
    \NomIntegrUno & \CiIntegrUno \\ \hline
    \NomIntegrDos & \CiIntegrDos  \\ \hline 
    \NomIntegrTres & \CiIntegrTres \\ \hline
    \NomIntegrCuatro & \CiIntegrCuatro \\ \hline
    \NomIntegrCinco & \CiIntegrCinco \\ \hline
  \end{tabular}\\[0.5cm]
\LARGE{\textbf{Docente}}\\	
\Large
\vspace{0.3cm}
\begin{tabular}{ | p{10cm} |}
    \hline
     \NomDocente \\ 
     \hline
  \end{tabular}
\end{flushleft}																		%%%
\end{minipage}		
																%%%
\begin{minipage}{0.52\textwidth}		
\vspace{-0.6cm}											%%%
\begin{flushright} \large															%%%
\emph{} \\																	%%%
													%%%
\end{flushright}																	%%%
\end{minipage}	
\begin{flushleft}
 	
\end{flushleft}
%%%
 		\flushleft{\textbf{}	}\\																		%%%								  						
\end{center}							 											
																					
%%%%%%%%%%%%%%%%%%%% TERMINA PORTADA %%%%%%%%%%%%%%%%%%%%%%%%%%%%%%%%
\newpage
\tableofcontents

\newpage
\section{Anexo X}
Por cada anexo a la documentación que deba ser agregado se tiene una sección como ésta. Los anexos son, en general, particularidades que deben ser documentadas como forma de complementar la documentación realizada. Además, se documentan particularidades que no corresponden a ninguna de las documentaciones exigidas y aquellas que sí corresponden pero no vale la pena realizar toda una documentación a tales efectos.

\subsection{Introducción}
En esta sección se introduce el problema.

\subsection{Solución}

\subsubsection{Explicación}
En esta sección se introduce la posible solución al problema.

\subsubsection{Diagramas}
Se muestran los diagramas necesarios para explicitar la solución.

\subsubsection{Alternativas}
Se explican alternativas de solución al problema.

\subsubsection{Referencias}
Se agregan referencias sobre el problema, de donde sacan la solución, detalles en otras documentaciones, etc.

\end{document}