\documentclass[a4paper,10pt]{article}
%\documentclass[a4paper,10pt]{scrartcl}
\usepackage[utf8]{inputenc}
\usepackage{hyperref}
\usepackage{url}
\usepackage[margin=1in]{geometry}
\usepackage{underscore}
\usepackage{amsfonts}
\usepackage{mathtools}

\title{Resumen del Teórico - Programación 3}
\author{Martin Pacheco}
\date{}

\pdfinfo{%
  /Title    ()
  /Author   ()
  /Creator  ()
  /Producer ()
  /Subject  ()
  /Keywords ()
}

\begin{document}


\maketitle
\section{Analisis del Algoritmo}
Consiste en determinar la \textbf{cantidad de operaciones} que realiza el algoritmo.\\ Para ello se tendrán en
cuenta sólo ciertas operaciones simples de los algoritmos, como suma, comparacion, asignacion, etc. \\
Es importante notar que cada operación tiene un costo asociado para el compilador que se utilice. \\\\
Según el contexto y la finalidad del análisis se necesitarán distintos niveles de precisión, por lo
tanto se verán tres posibles puntos de vista para el análisis:
\begin{itemize}
  \item Contar las operaciones simples con sus costos.
  \item Contar las operaciones simples sin tener en cuenta costos.
  \item Contar sólo una operación predefinida y/o básica.
\end{itemize}
\section{Algoritmo}
\begin{enumerate}
  \item Secuencia finita de pasos.
  \item Cada paso correctamente definido.
  \item Cada paso debe ejecutarse en un tiempo finito.
  \item Termina en algún momento.
  \item Devuelve el resultado esperado.
  \item Tiene un dominio de definición.
\end{enumerate}
\section{Dominio de Definicion}
Un dominio de definición D, es el conjunto de TODAS las entradas posibles.\\
Por otro lado el dominio de definición es el conjunto de entradas para la cual fue diseñado el
algoritmo y en ese sentido asegura la correctitud del mismo. \\ El mismo algoritmo con otro
dominio de definición, puede no ser correcto. \\\\
\textbf{Por ejemplo:} si está diseñado para trabajar sobre números
naturales no tiene porqué funcionar si se pretende usar sobre números reales. \\\\
\textbf{Notacion:} \\ En general el dominio de definición se expresa en función del tamaño de la entrada.
\begin{itemize}
  \item $D_n = \{ E / E \in D, \textbf{Tamaño}(E) = n \} $
  \item Se denotará \textbf{T(E}) al costo (individual) de la entrada E.
\end{itemize}
\section{Analisis}
Se busca determinar, analizando un algoritmo dado, el costo del mismo expresándolo como una
función T del tamaño n de la entrada: T(n).
\subsection{Peor Caso - (W)orst}
Se obtendrá el mayor costo posible. \\\\
$\begin{array}{ll}
T_W(n) = max\{T(E) / E \in D_n \} & \textsc{// costo en el peor caso} \\
\{E / T(E) = T_W(n), E \in D_n \} & \textsc{// conjunto de entradas que son peor caso}
\end{array}$

\subsection{Mejor Caso - (B)etter}
Se tendrá el menor costo posible. \\\\
$\begin{array}{ll}
T_B(n) = min\{T(E) / E \in D_n \} & \textsc{// costo en el mejor caso} \\
\{E / T(E) = T_B(n), E \in D_n \} & \textsc{// conjunto de entradas que son en el mejor caso}
\end{array}$

\subsection{Caso Promedio - (A)verage}
Costo en un caso basado en ciertas hipotesis y probabilidades. \\
$\text{Promedio de TODAS las } T(E), E \in D_n$.\\\\
$\begin{array}{ll}
T_A(n) = \displaystyle\sum_{{E \in D_n}} p(E) * T(E) & \textsc{// costo en caso promedio} \\
\{E / T(E) = T_A(n), E \in D_n\} & \textsc{// conjunto de entradas en caso promedio}\\
\end{array}$\\\\
donde $p(E)$ es la probabilidad que se de la entrada E, \textbf{probabilidad EXPERIMENTAL}.\\

\subsection{Ejemplo: Find}
Encontrar un elemento x dado, en una secuencia A dada, (puede considerarse que todos los
elementos de la secuencia son distintos), si el elemento está devuelve la posición, sino devuelve el
valor -1.
\begin{verbatim}
//a array of int [0...N-1]
// n cant elementos;
int find(int* A, int n, int x){
  int i = 0;
  while (i<n) && (A[i]!=x){
    i++
  }
  if (i<n)
    return i;
  else
    return -1;
}
\end{verbatim}
\textbf{\large Análisis} - considerando sólo \textbf{operación básica}.
\subsubsection{Mejor Caso}
Secuencia donde x está en el 1er lugar.\\
$T_B(n) = 1$
\subsubsection{Peor Caso}
Hay 2 configuraciones en las que se da el peor caso, estas son cuando:
\begin{itemize}
  \item X no está en A
  \item X es el último elemento de A
\end{itemize}
$T_w(n) = n$
\section{Caso Promedio}
\textbf{Asumimos} que si $x \in A$, todas las posiciones son equiprobables.\\
Se parte de y se debe obtener resultados para la expresión: $T_A(n) = \displaystyle\sum_{{E \in D_n}} p(E) * T(E)$\\
esta expresión debe ser transformada para que resulte práctica.\\\\
\textbf{Observación:}
$D_n$ es el conjunto de todas las secuencias de largo n.\\
En la práctica cabe notar que el algoritmo dado se comporta (ejecuta) de la misma forma para todas las secuencias que tengan a x en un lugar específico i: No interesa el resto de los elementos que compongan la secuencia.\\Si x está en el lugar i-ésimo se realizan i comparaciones y el algoritmo termina. Sucede algo similar si el elemento no está en la secuencia.\\\\
\textbf{Caso 1:} $x \in A$.\\
En base a la observación anterior, es posible subdividir $D_n$ en subconjuntos $D_{n_i} \subset D_n$ , donde $D_{n_i}$ está formado por las secuencias que tienen a x en la posición i-ésima, $1 \leq i \leq n$. \\
Resulta posible modificar el enfoque del problema y considerar que se tienen sólo n entradas $E_i$ , $1 \leq i \leq n$. Serán secuencias que sólo difieren en la posición del elemento x.\\\\
En ese caso se tendrá:\\
$p(E_i) = \frac{1}{n}$, debido a la hipótesis de equiprobabilidad de ubicación de \textbf{x en A}.\\\\
$T(E_i) = i$\\\\
$T_A(n) = \displaystyle\sum_{i=1}^{n} p(E_i) * T(E_i) = \sum_{i=1}^n \frac{1}{n} * i = \frac{1}{n} * \sum_{i=1}^n i = \frac{1}{n} * \frac{n(n+1)}{2} \Rightarrow T_A(n) = \frac{(n+1)}{2}$\\\\\\
\textbf{Caso 2:} x puede no estar en A.\\
Probabilidad de que $x \in A = q$.\\
Analogamente al \textbf{Caso 1}:
\begin{enumerate}
  \item Si $1 \leq i \leq n$, $E_i$ representa las instancias $x \in A$ en el lugar i.
  \item $E_{n+1}$ representa a las instancias de $x\notin A$.
\end{enumerate}
Se tiene:
\begin{itemize}
  \item $p(E_i) = \frac{q}{n}, \text{ si } 1\leq i \leq n$
  \item $p(E_i) = 1-q$
  \item $T(E_i) = i , \text{ si } 1\leq i \leq n$
  \item $T(E_{n+1}) = n$
\end{itemize}
$T_A(n) = \displaystyle\sum_{i=1}^{n+1} p(E_i) * T(E_i) = (1-q) * n + \sum_{i=1}^n i * \frac{q}{n} = (1-q) * n + \frac{q}{n} * \sum_{i=1}^n i$\\
$\Rightarrow T_A(n) = \displaystyle\frac{q}{n} * n * \frac{n+1}{2} + (1-q) * n$\\
$\Rightarrow T_A(n) = \displaystyle q * \frac{n+1}{2} + (1-q) * n$\\\\
\textbf{Notar:}
\begin{itemize}
  \item \textbf{si q = 1}, estamos en el caso 1 y el resultado es idéntico al obtenido antes.
  \item \textbf{si q = 0}, entonces $x \notin A$ y el costo es n como en el peor caso visto.
\end{itemize}
\section{Comportamiento Asintotico}
En muchos casos resultará de interés investigar el costo de un algoritmo para entradas de tamaño
n arbitrariamente grande, es decir, \textbf{cuando n tiende a infinito}.\\
Esto implica que hay términos del costo que se pueden despreciar ya que su aporte es ínfimo.\\\\
En definitiva interesa determinar la \textbf{tasa de crecimiento de T(n)}.\\
La tasa de crecimiento exponencial implica una notable mayor sensibilidad al tamaño de la
entrada y menor sensibilidad a la velocidad de procesamiento.

\section{Nocion Informal}
T(n) \textbf{es del orden} de f(n) si T(n) es acotada por un múltiplo real positivo de f(n).
\end{document}
