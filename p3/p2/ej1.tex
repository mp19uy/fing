\section*{Ejercicio 1}
\subsection*{Parte 1)}
\begin{itemize}
  \item (1) operacion = aplicada a: sum y 0.
  \item (1) OP = aplicada a: i y 1. Al entrar por primera vez al for.
  \item Suma desde 1 a n:
  \begin{itemize}
    \item (1) OP <= aplicada a: i y n.
    \item (1) OP ++ aplicada a: sum.
    \item (1) OP ++ aplicada a: i.
  \end{itemize}
\item (1) OP <= aplicada a: i y n. Al salir del for.
\end{itemize}
\line(1,0){200}
\subsubsection*{Exacto}
\begin{align*}
&T(n) = 1+1+\sum_{i=1}^{n}{(1+1+1)} + 1\\
&T(n) = 3 + n \cdot 3\\ 
&\mathbf{T(n) = 3(n+1)}
\end{align*}
\line(1,0){200}
\subsubsection*{En base a Op: sum++}
\begin{align*}
&T(n) = \sum_{i=1}^{n}{(1)}\\
&T(n) = n \cdot 1\\
&\mathbf{T(n) = n}
\end{align*}
\line(1,0){200}
\subsubsection*{Orden}
Como el Tiempo Exacto es: $\displaystyle 3n + 3$\\
Entonces $\mathbf{T(n) = n}$\\
\pagebreak
\subsection*{Parte 2)}
\begin{itemize}
  \item (1) operacion = aplicada a: sum y 0.
  \item (1) OP = aplicada a: i y 1. Al entrar por primera vez al for.
  \item Suma desde 1 a n:
  \begin{itemize}
    \item (1) OP <= aplicada a: i y n.
    \item (1) OP = aplicada a: j y 0. Al entrar por primera vez al for.
    \item Suma desde 1 a n:
      \begin{itemize}
        \item (1) OP <= aplicada a: j y n.
        \item (1) OP ++ aplicada a: sum.
        \item (1) OP ++ aplicada a: j.
      \end{itemize}
    \item (1) OP <= aplicada a: j y n. Al salir del for.
  \end{itemize}
\item (1) OP <= aplicada a: i y n. Al salir del for.
\end{itemize}
\line(1,0){200}
\subsubsection*{Exacto}
\begin{align*}
&T(n) = 1+1+\sum_{i=1}^{n}\left(1+1+\sum_{j=1}^{n}{(1+1+1)}+1\right) + 1\\
&T(n) = 3 + n(3 + 3n)\\ 
&\mathbf{T(n) = 3(n^2+n+1)}
\end{align*}
\line(1,0){200}
\subsubsection*{En base a Op: sum++}
\begin{align*}
&T(n) = \sum_{i=1}^{n}\left(\sum_{j=1}^{n}(1)\right)\\
&T(n) = n \cdot n \cdot 1\\
&\mathbf{T(n) = n^2}
\end{align*}
\line(1,0){200}
\subsubsection*{Orden}
Como el Tiempo Exacto es: $\displaystyle 3n^2 + 3n + 3$\\
Entonces $\mathbf{T(n) = n^2}$
\pagebreak
\subsection*{Parte 3)}
\begin{itemize}
  \item (1) operacion = aplicada a: sum y 0.
  \item (1) OP = aplicada a: i y 1. Al entrar por primera vez al for.
  \item Suma desde 1 a n:
  \begin{itemize}
    \item (1) OP <= aplicada a: i y n.
    \item (1) OP = aplicada a: j y 0. Al entrar por primera vez al for.
    \item Suma desde 1 a $n^2$:
      \begin{itemize}
        \item (1) OP * aplicada a n.
        \item (1) OP <= aplicada a: j y n.
        \item (1) OP ++ aplicada a: sum.
        \item (1) OP ++ aplicada a: j.
      \end{itemize}
    \item (1) OP * aplicada a n. Al salir del for.
    \item (1) OP <= aplicada a: j y n. Al salir del for.
  \end{itemize}
\item (1) OP <= aplicada a: i y n. Al salir del for.
\end{itemize}
\line(1,0){200}
\subsubsection*{Exacto}
\begin{align*}
&T(n) = 1+1+\sum_{i=1}^{n}\left(1+1+\sum_{j=1}^{n^2}{(1+1+1+1)}+1+1\right) + 1\\
&T(n) = 3 + n(4 + 4n^2)\\ 
&\mathbf{T(n) = 4(n^3+n)+3}
\end{align*}
\line(1,0){200}
\subsubsection*{En base a Op: sum++}
\begin{align*}
&T(n) = \sum_{i=1}^{n}\left(\sum_{j=1}^{n^2}(1)\right)\\
&T(n) = n \cdot n^2 \cdot 1\\
&\mathbf{T(n) = n^3}
\end{align*}
\line(1,0){200}
\subsubsection*{Orden}
Como el Tiempo Exacto es: $\displaystyle 4n^3 + 4n + 3$\\
Entonces $\mathbf{T(n) = n^3}$
