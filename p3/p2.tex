%%%%%%%%%%%%%%%%%%%%%%%%%%%%%%%%%%%%%%%%%%%%%%%%%%%%%%%%%%%%%%%%%%%%%%%%%%%%%
% Practico 2 - Programacion 3
%%%%%%%%%%%%%%%%%%%%%%%%%%%%%%%%%%%%%%%%%%%%%%%%%%%%%%%%%%%%%%%%%%%%%%%%%%%%%

\documentclass[letterpaper, fleqn]{report}

% This first part of the file is called the PREAMBLE. It includes
% customizations and command definitions. The preamble is everything
% between \documentclass and \begin{document}.
\newcommand{\margin}{2cm}
\usepackage[top=\margin,right=\margin,left=\margin,bottom=\margin]{geometry}


%%%%%%%%%%%%%%%%%%%%%%%%%%%%%%%%%%%%%%%%%%%%%%%%%%%
% Another way to Set the margins
%%%%%%%%%%%%%%%%%%%%%%%%%%%%%%%%%%%%%%%%%%%%%%%%%%%
%\usepackage[margin=2cm]{geometry}  % set the %margins to 2cm on all sides
%%%%%%%%%%%%%%%%%%%%%%%%%%%%%%%%%%%%%%%%%%%%%%%%%%%

% to use urls
\usepackage{hyperref}

% to include figures
\usepackage{graphicx}

% great math stuff
\usepackage{amsmath}

% for blackboard bold, etc
\usepackage{amsfonts}

% better theorem environments
\usepackage{amsthm}

% para tildes y eñes y otras yerbas
\usepackage[utf8]{inputenc}
\usepackage[T1]{fontenc}
\usepackage{lmodern}
  
% para links y emails
\usepackage{hyperref}

% Use fancyhdr to define our own headers
\usepackage{fancyhdr}
\setlength{\headheight}{25pt} 

% Keeps LaTeX happy, takes care of some warnings
\pagestyle{fancy}

% Multi rows
\usepackage{multirow}

% caracteres
\usepackage{textcomp}
% various theorems, numbered by section
\newtheorem{thm}{Theorem}[section]
\newtheorem{lem}[thm]{Lemma}
\newtheorem{prop}[thm]{Proposition}
\newtheorem{cor}[thm]{Corollary}
\newtheorem{conj}[thm]{Conjecture}

\DeclareMathOperator{\id}{id}

% for bolding symbols
\newcommand{\bd}[1]{\mathbf{#1}}

% for Real numbers
\newcommand{\RR}{\mathbb{R}}

% for Integers
\newcommand{\ZZ}{\mathbb{Z}}
\newcommand{\col}[1]{\left[\begin{matrix} #1 \end{matrix} \right]}
\newcommand{\comb}[2]{\binom{#1^2 + #2^2}{#1+#2}}

% Definition of the title
\title{Resolución del \practico\ \\de \materia}

% Definition of the author
\author{Martín Pacheco\\Estudiante de Ingeniería en Computación\\Facultad de Ingeniería, UDELAR, Montevideo, Uruguay}
\date{\anio}
% Definitions to fill the header with
%%%%%%%%%%%%%%%%%%%%%%%%%%%%%%%%%%%%%%%%%%%%%%%%%%%%%%%%%%%%%%%%%%%%%%%%%%%%%%%%
\newcommand{\materia}{Programación 3}
\newcommand{\practico}{Práctico 2}
\newcommand{\name}{Martín Pacheco}
\newcommand{\facultad}{Facultad de Ingeniería}
\newcommand{\semestre}{Segundo Semestre}
\newcommand{\anio}{2014}
%%%%%%%%%%%%%%%%%%%%%%%%%%%%%%%%%%%%%%%%%%%%%%%%%%%%%%%%%%%%%%%%%%%%%%%%%%%%%%%%

% Redefinition of commands
\newcommand{\sectionh}[1]{\line(1,0){500}\section*{#1}}
% Now custimize the header. Make the text bold.
% We'll get something like:
%
% 123456789 LaTeX 101
% J. Random Student Assignment N Today's Date
% --------------------------------------------------
%
% This layout is pretty simple, and should be enough for an assignment
% If you want more, you can consult the documentation
% http://www.ctan.org/tex-archive/macros/latex/contrib/fancyhdr/fancyhdr.pdf
\lhead{\textbf{\facultad\\ \name}}
\chead{\textbf{\materia\\ \practico}}
\rhead{\textbf{\semestre\\ \anio}}

% Here is an example for customising the numbering
% It changes the first level of numbering to bolded (a), (b), (c), etc
%\renewcommand{\theenumi}{\textbf{(\alph{enumi})}}
%\renewcommand{\labelenumi}{\theenumi}
% Other options to play with are to change \theenumii, \labelenumii, and enumii for the second level of nesting,
% and so on to \theenumiv, \labelenumiv, and enumiv for the fourth level of nesting.
% The possible formats are \arabic (1, 2...), \alph (a, b...), \Alph (A, B...), \roman (i, ii...), and \Roman (I, II...)
\setlength{\mathindent}{0cm}

% Here begin the document
\begin{document}

% Print the title of the document
\maketitle
\def\abstractname{ Sobre este documento...}
\begin{abstract}
  Son resoluciones del \practico\ del curso de \materia\ dictado en el \semestre\ de \anio, en la \facultad, UDELAR, Montevideo, Uruguay. 
  \\\\
  Estas resoluciones son en parte propias, en parte de otros estudiantes de la facultad y en parte hechas a partir de otras resoluciones de prácticos de años anteriores y material obtenido en internet con ejercicios similares resueltos.
  \\\\
  La motivación de escribirlo \LaTeX\ ha sido por mi interés en aprender el lenguaje y para poder proporcionar de una manera más ordenada y legible, las resoluciones del prácticos a otros estudiantes que se puedan beneficiar con el uso de este material.
  \\\\
  Por sugerencias, comentarios y errores sobre este material, enviar un email a \href{mailto:mp19uy@gmail.com}{mp19uy@gmail.com} .
\end{abstract}
\section*{Ejercicio 1}
\subsection*{Parte 1)}
\begin{itemize}
  \item (1) operacion = aplicada a: sum y 0.
  \item (1) OP = aplicada a: i y 1. Al entrar por primera vez al for.
  \item Suma desde 1 a n:
  \begin{itemize}
    \item (1) OP <= aplicada a: i y n.
    \item (1) OP ++ aplicada a: sum.
    \item (1) OP ++ aplicada a: i.
  \end{itemize}
\item (1) OP <= aplicada a: i y n. Al salir del for.
\end{itemize}
\line(1,0){200}
\subsubsection*{Exacto}
\begin{align*}
&T(n) = 1+1+\sum_{i=1}^{n}{(1+1+1)} + 1\\
&T(n) = 3 + n \cdot 3\\ 
&\mathbf{T(n) = 3(n+1)}
\end{align*}
\line(1,0){200}
\subsubsection*{En base a Op: sum++}
\begin{align*}
&T(n) = \sum_{i=1}^{n}{(1)}\\
&T(n) = n \cdot 1\\
&\mathbf{T(n) = n}
\end{align*}
\line(1,0){200}
\subsubsection*{Orden}
Como el Tiempo Exacto es: $\displaystyle 3n + 3$\\
Entonces $\mathbf{T(n) = n}$\\
\pagebreak
\subsection*{Parte 2)}
\begin{itemize}
  \item (1) operacion = aplicada a: sum y 0.
  \item (1) OP = aplicada a: i y 1. Al entrar por primera vez al for.
  \item Suma desde 1 a n:
  \begin{itemize}
    \item (1) OP <= aplicada a: i y n.
    \item (1) OP = aplicada a: j y 0. Al entrar por primera vez al for.
    \item Suma desde 1 a n:
      \begin{itemize}
        \item (1) OP <= aplicada a: j y n.
        \item (1) OP ++ aplicada a: sum.
        \item (1) OP ++ aplicada a: j.
      \end{itemize}
    \item (1) OP <= aplicada a: j y n. Al salir del for.
  \end{itemize}
\item (1) OP <= aplicada a: i y n. Al salir del for.
\end{itemize}
\line(1,0){200}
\subsubsection*{Exacto}
\begin{align*}
&T(n) = 1+1+\sum_{i=1}^{n}\left(1+1+\sum_{j=1}^{n}{(1+1+1)}+1\right) + 1\\
&T(n) = 3 + n(3 + 3n)\\ 
&\mathbf{T(n) = 3(n^2+n+1)}
\end{align*}
\line(1,0){200}
\subsubsection*{En base a Op: sum++}
\begin{align*}
&T(n) = \sum_{i=1}^{n}\left(\sum_{j=1}^{n}(1)\right)\\
&T(n) = n \cdot n \cdot 1\\
&\mathbf{T(n) = n^2}
\end{align*}
\line(1,0){200}
\subsubsection*{Orden}
Como el Tiempo Exacto es: $\displaystyle 3n^2 + 3n + 3$\\
Entonces $\mathbf{T(n) = n^2}$
\pagebreak
\subsection*{Parte 3)}
\begin{itemize}
  \item (1) operacion = aplicada a: sum y 0.
  \item (1) OP = aplicada a: i y 1. Al entrar por primera vez al for.
  \item Suma desde 1 a n:
  \begin{itemize}
    \item (1) OP <= aplicada a: i y n.
    \item (1) OP = aplicada a: j y 0. Al entrar por primera vez al for.
    \item Suma desde 1 a $n^2$:
      \begin{itemize}
        \item (1) OP * aplicada a n.
        \item (1) OP <= aplicada a: j y n.
        \item (1) OP ++ aplicada a: sum.
        \item (1) OP ++ aplicada a: j.
      \end{itemize}
    \item (1) OP * aplicada a n. Al salir del for.
    \item (1) OP <= aplicada a: j y n. Al salir del for.
  \end{itemize}
\item (1) OP <= aplicada a: i y n. Al salir del for.
\end{itemize}
\line(1,0){200}
\subsubsection*{Exacto}
\begin{align*}
&T(n) = 1+1+\sum_{i=1}^{n}\left(1+1+\sum_{j=1}^{n^2}{(1+1+1+1)}+1+1\right) + 1\\
&T(n) = 3 + n(4 + 4n^2)\\ 
&\mathbf{T(n) = 4(n^3+n)+3}
\end{align*}
\line(1,0){200}
\subsubsection*{En base a Op: sum++}
\begin{align*}
&T(n) = \sum_{i=1}^{n}\left(\sum_{j=1}^{n^2}(1)\right)\\
&T(n) = n \cdot n^2 \cdot 1\\
&\mathbf{T(n) = n^3}
\end{align*}
\line(1,0){200}
\subsubsection*{Orden}
Como el Tiempo Exacto es: $\displaystyle 4n^3 + 4n + 3$\\
Entonces $\mathbf{T(n) = n^3}$

\pagebreak
\subsection*{Parte 4)}
\begin{itemize}
  \item (1) operacion = aplicada a: sum y 0.
  \item (1) OP = aplicada a: i y 1. Al entrar por primera vez al for.
  \item Suma desde 1 a n:
  \begin{itemize}
    \item (1) OP <= aplicada a: i y n.
    \item (1) OP = aplicada a: j y 0. Al entrar por primera vez al for.
    \item Suma desde 1 a $i$:
      \begin{itemize}
        \item (1) OP <= aplicada a: j y i.
        \item (1) OP ++ aplicada a: sum.
        \item (1) OP ++ aplicada a: j.
      \end{itemize}
    \item (1) OP <= aplicada a: j y n. Al salir del for.
  \end{itemize}
\item (1) OP <= aplicada a: i y n. Al salir del for.
\end{itemize}
\line(1,0){200}
\subsubsection*{Exacto}
\begin{align*}
&T(n) = 1+1+\sum_{i=1}^{n}\left(1+1+\sum_{j=1}^{i}{(1+1+1)}+1\right) + 1\\
&T(n) = 3 + n(4 + 4n^2)\\ 
&\mathbf{T(n) = 4(n^3+n)+3}
\end{align*}
\line(1,0){200}
\subsubsection*{En base a Op: sum++}
\begin{align*}
&T(n) = \sum_{i=1}^{n}\left(\sum_{j=1}^{n^2}(1)\right)\\
&T(n) = n \cdot n^2 \cdot 1\\
&\mathbf{T(n) = n^3}
\end{align*}
\line(1,0){200}
\subsubsection*{Orden}
Como el Tiempo Exacto es: $\displaystyle 4n^3 + 4n + 3$\\
Entonces $\mathbf{T(n) = n^3}$

\end{document}
